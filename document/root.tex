\documentclass[11pt,a4paper]{article}
\usepackage[T1]{fontenc}
\usepackage{amsmath,amssymb}
\usepackage{isabelle,isabellesym}

\usepackage[ngerman,american]{babel}
\usepackage[%
  backend=biber,
  maxbibnames=99,
  url=false,
  sorting=none,
  datamodel=software,
  % style=alphabetic,
  % maxnames=4,
  % minnames=3,
  maxbibnames=99,
  %giveninits,
  %uniquename=init
  ]{biblatex}

% this should be the last package used
\usepackage{pdfsetup}

% urls in roman style, theory text in math-similar italics
\urlstyle{rm}
\isabellestyle{it}


\begin{document}

\title{Topological Groups}
\author{Niklas Krofta}
\maketitle

\begin{abstract}
Topological groups are blends of groups and topological spaces with the property that the multiplication and inversion operations are continuous functions. 
They frequently occur in mathematics and physics, e.g. in the form of Lie groups. We formalize the theory of topological groups on top of 
HOL-Algebra and HOL-Analysis. Topological groups are defined via a locale. We also introduce a set-based notion of uniform spaces in order to define
the uniform structures of topological groups. The most notable formalized result is the Birkhoff-Kakutani theorem which characterizes 
metrizable topological groups. Our formalization also defines the important matrix groups $\mathrm{GL}_n(\mathbb{R})$, $\mathrm{SL}_n(\mathbb{R})$,
$\mathrm{O}_n$, $\mathrm{SO}_n$ and proves them to be topological groups. \newline

The formalized results and proofs have been taken from the textbooks of Arhangel’skii and Tkachenko \cite{topological_groups}, Bump \cite{lie_groups}
and James \cite{topological_and_uniform_spaces}. These lecture notes \cite{prem} have also been helpful.
\end{abstract}

\tableofcontents

% include generated text of all theories
\input{session}

\bibliographystyle{abbrv}
\bibliography{root}

\end{document}
\endinput
